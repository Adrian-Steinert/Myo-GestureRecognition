\documentclass[]{scrartcl}

\usepackage[utf8]{inputenc}

\setlength{\parindent}{0pt}

%opening
\title{Vorhabensbeschreibung -- Bachelorarbeit}
\author{Adrian Steinert}

\begin{document}

\maketitle

Das Myo Armband von Thalmic Labs ist ein Eingabegerät für verschiedene Betriebssysteme, welches Hand- und Armgesten des Benutzers aufnimmt, um diese dann mittels vordefinierter oder eigener Programme als Eingabe zu interpretieren. Die Gesten werden mit Hilfe von Oberflächensensoren für die Elektromyographie (EMG) der Unterarmmuskulatur und einer inertialen Messeinheit (IMU) erfasst, welche Lage und Beschleunigung des Armbandes berechnet. Seitens des Herstellers ist das Armband in der Lage, fünf vordefinierte EMG-Gesten zu erkennen. Diese können dann programmatisch mit IMU-Gesten kombiniert werden, um so neue Gesten zu ermöglichen.\\

In dieser Bachelorarbeit sollen Hand- und Armgesten mit dem Myo Armband aufgezeichnet und mit Machine Learning Algorithmen gelernt werden. Hierbei wird festgestellt, ob eine Gestenerkennung anhand dieser Daten möglich ist. Für das Training wird jede Geste separat als eine Sequenz aufgezeichnet. Beginn und Ende einer Gestensequenz werden durch den Anwender festgelegt.\\

Die automatische Erfassung einer Gestensequenz kann als Erweiterung ermöglicht werden. Sofern dies gelingt, wird das Ganze an einem Anwendungsbeispiel veranschaulicht. Bei diesem Beispiel soll ein NAO Roboter auf erkannte Gesten reagieren.



\end{document}
